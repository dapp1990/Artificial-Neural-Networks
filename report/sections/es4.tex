\subsection{Stacked autoencoders}
After analyzing \textbf{DigitClassification.m} script, several experiments were run. Tables \ref{table_sec_4_1} and \ref{table_sec_4_2} show the results of different models with different hyperparameters.
\bigbreak
All the models could beat the original one (second column in table \ref{table_sec_4_1}) but the model with third hidden layers (seventh column in table \ref{table_sec_4_1}) before the fined-tuning. However after the fined-tuning, the DeepNets could not overcome the original setting. Even the best model (fifth column in table \ref{table_sec_4_1}) was not able to perform better.
\bigbreak
Three interesting remark are noticed. The model with best results after fined-tuning took around 7 minutes per run. That is a significant amount of time compared with the other models, even the second best model before fined-tuning (sixth column in table \ref{table_sec_4_1}) took around 3 minutes per run. This is a massive impact in the time complexity of the overall model where the gain was relatively small (1.03). Is really it a good tradeoff between time and accuracy?. It precisely depends on the application, for instance in applications where the accuracy is highly important such as the health-care sector.
\bigbreak
However, this tradeoff between time and accuracy can be easily solved using the fined-tuning phase. Clearly the results shown that the fined-tuning phase improve all models, even the worst model that used 3 hidden layers. It was completely not expected that the accuracy of the models improved to beat even the FFNN. Fined-tuning phase has shown that is an important phase in order to improve, even in the worst cases, the accuracy of the model.
\bigbreak
Finally, after fined-tuning all the models beat the FFNN with 1 and 2 hidden layers. However the overall performance are not bad whatsoever. Reaching an average of 96.94 and 96.45, FFNN has shown that it is a very powerful as well as a simple model. As aforementioned, FFNN can be implemented in many applications that do not require heavy critical accuracy.
\begin{table}[!htbp]
\centering
\caption{Results of DeepNet before fined-tuning. (\# layer, max epochs, hidden units)}
\label{table_sec_4_1}
\medbreak
\begin{tabular}{c|c|c|c|c|c|c}
 & \pbox{4cm}{1, 400, 100\\2, 100, 50} & \pbox{4cm}{1, 400, 100\\2, 400, 50} & \pbox{4cm}{1, 100, 100\\2, 400, 50} & \pbox{4cm}{1, 400, 400\\2, 100, 200} & \pbox{4cm}{1, 400, 200\\2, 100, 100} & \pbox{4cm}{1, 400, 100\\2, 100, 50\\3, 50, 25}\\\hline
1 & 86.72 & 94.16 & 91.28 & 99.18 & 98.24 & 28.84\\\hline
2 & 81.18 & 94.70 & 93.12 & 99.02 & 97.76 & 29.80\\\hline
3 & 89.16 & 95.36 & 93.70 & 98.32 & 97.84 & 13.68\\\hline
4 & 78.66 & 93.88 & 92.26 & 98.82 & 97.92 & 17.18\\\hline
5 & 82.48 & 92.62 & 91.94 & 98.90 & 97.34 & 17.18\\\hline
\textbf{Avg} & \textbf{83.64} & \textbf{94.14} & \textbf{92.46} & \textbf{98.85} & \textbf{97.82} & \textbf{21.34} 
\end{tabular}
\end{table}


\begin{table}[!htbp]
\centering
\caption{Results of DeepNet after fined-tuning. (\# layer, max epochs, hidden units)}
\label{table_sec_4_2}
\medbreak
\begin{tabular}{c|c|c|c|c|c|c}
 & \pbox{4cm}{1, 400, 100\\2, 100, 50} & \pbox{4cm}{1, 400, 100\\2, 400, 50} & \pbox{4cm}{1, 100, 100\\2, 400, 50} & \pbox{4cm}{1, 400, 400\\2, 100, 200} & \pbox{4cm}{1, 400, 200\\2, 100, 100} & \pbox{4cm}{1, 400, 100\\2, 100, 50\\3, 50, 25}\\\hline
1 & 99.68 & 98.96 & 99.52 & 99.48  &98.86 & 99.24\\\hline
2 & 99.76 & 98.92& 99.52 & 99.56 & 98.88 & 99.30\\\hline
3 & 99.80 & 99.06& 99.52 & 99.10 & 98.36 & 99.32\\\hline
4 & 99.78 & 98.72& 99.54 & 99.20 & 98.04 & 99.32 \\\hline
5 & 99.72 & 98.86 & 99.72 & 99.32 & 99.00 & 99.32\\\hline
\textbf{Avg} & \textbf{99.75} & \textbf{98.90} & \textbf{99.56} & \textbf{99.33} & \textbf{98.64} & \textbf{99.30} 
\end{tabular}
\end{table}


\begin{table}[!htbp]
\centering
\caption{Results using a FFNN.}
\label{table_sec_4_3}
\medbreak
\begin{tabular}{c|c|c}
1 hidden layer & 2 hidden layers \\\hline
1 & 97.02 & 95.84 \\\hline
2 & 96.64 & 97.52 \\\hline
3 & 96.28 & 97.26  \\\hline
4 & 96.28 & 97.26 \\\hline
5 & 97.80 & 94.42 \\\hline
\textbf{Avg} & \textbf{96.94} & \textbf{96.45}
\end{tabular}
\end{table}


\subsection{Convolutional neural networks}
\textbf{What do these weights represent?}
\smallbreak
Those weights represent the stack of filtered features (in this specific application, pixeles). 
\bigbreak
\textbf{What is the dimension of the input at the start of layer 6 and why?}
\smallbreak
The previous layer 5 is a MaxPooing layer. It was found that the \textit{PoolSize} was set to 3x3 pixels and \textit{Stride} to 2x2. For each filtered image in the stack, it reduces the input size from 11x11x3 to 5x5x3. According with the MatLab documentation \cite{matlab_1}, the attribute \textit{NumChannels} represents the feature maps. In this documentation, they state that this input value corresponds to the number of filters in the previous convolutional layer. The number of filters of layer 2 is 96. Hence, the input must take into account the shrunk filtered pixeles by the numbers of filters. The input for layer 6 is of 4 dimensions [5,5,3,96], the same dimensionality than the output of layer 2. An interesting observation is that actually the \textit{NumChannels}  parameter of layer 6 is a vector of [48,48] and not an integer value of 96. According with \cite{matlab_1} this cannot be set manually as a vector, it must be a integer value. It was assumed that the filters are split due to the CrossChannelNormalization layer and Matlab internally accepts that type of input for Layer 6.
\bigbreak
\textbf{What is the final dimension of the problem? How does this compare with the initial dimension?}
\smallbreak
According with the ClassificationOutput layer, it contains a single vector (1 dimension) with 1000 of neurons. The initial dimension was [227,227,3] which gives us a total of 154,587 elements. The reduction is significant, the initial input was cut down roughly 155 times.



